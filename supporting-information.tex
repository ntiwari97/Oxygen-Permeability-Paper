\documentclass[12pt]{article}
\usepackage[utf8]{inputenc}

\usepackage{amsmath}
\usepackage{setspace}

\begin{document}
\doublespacing
\section*{Supporting Information}
Water content was measured as a function of RH for both PFSA and HOPI. To prepare thin films, the stock dispersions were diluted to 2 wt. \%, with the final solvent composition 50\% water by mass, and the balance nPA. Silicon (Si) substrates were prepared by rinsing with water and isopropanol (IPA), drying with N$_2$, and cleaning with plasma. Immediately after cleaning, the dispersions were spin-casted onto the substrate at 3000 rpm for 1 minute. Films were annealed at 150°C under vacuum for 1 hour, and were exposed to nitrogen at various RHs. During the first conditioning cycle, RH was varied from 0 to 100, then back to 0\%. During the absorption cycle, RH was increased in gradual increments, and the thickness of the film at each increment was measured. The degree of film swelling, $\Delta L$, is the difference in thickness between the film at the end of the conditioning cycle, $L_0$, (when it has been exposed to 0\% RH nitrogen) and the thickness of the film at the current RH increment in the absorption cycle. The swelling data was converted to water content $\lambda$, the number of moles of water per mole of ionomer sulfonate moiety, by assuming that swelling was strictly one dimensional, and that there was no significant excess free volume in the film. Water content is given by:

\begin{equation*}
\lambda = \frac{mol H_2O}{mol SO_3^-} = \frac{\Delta L}{L_0}\frac{\rho_W/MW_W}{\rho_I/EW}    
\end{equation*}

Where W and I denote water and ionomer, respectively, $\rho$ is dry density, MW is molecular weight and EW is the equivalent weight of the ionomer (in this case, 850 for PFSA and 872 for HOPI). 
\end{document}
